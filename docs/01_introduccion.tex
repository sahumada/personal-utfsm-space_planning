\section{Introducción\label{sec:introduccion}}

Los problemas de Carga de Containers (Container Loading Problem) y
los problemas de Empaquetamiento (Bin Packing Problem) son ampliamente
usados en áreas como Diseño de circuitos VLSI, Carga de pallets, Carga
de camiones, etc. \\

En general, los problemas de Space Planning 3D han acaparado la atención
de muchos investigadores en el último tiempo, donde cada uno ha desarrollado
diversas técnicas de resolución que hacen el problema más interesante. Entre
las variantes de este problema están aquellos que sólo utilizan un tipo de cubos;
aquellos denominados ``guillotinables'' o que intentan realizar un orden tal que
es posible cortar de forma perfecta el contenedor con un plano imaginario; aquellos
en que se deben depositar los objetos de arriba hacia abajo, etc. \\

Los problemas de empaquetamiento (Bin Packing Problem) describen la
consistencia de combinaciones de pequeños objetos geométricos en grandes
superficies. En el caso de problemas del tipo Bin Packing las
superficies, $S$, son definidas vacias y se necesita posicionar una
cierta cantidad, $n$, de objetos en su interior. Los objetos no pueden
superponerse ni pueden quedar fuera de la superficie definida, estos
objetos no podrán rotarse, debido a que en el caso 3D la cantidad de
posibles posiciones de un objeto dificulta la resolución del problema.
Además, todos los objetos deben ser puestos en su totalidad en $S$. \\

En este trabajo se presenta el problema combinatorial, en la cual se
trata de ubicar objetos de ciertas dimensiones dadas sobre un volumen
determinado sin que ellos salgan de dicho volumen o se traslapen. \\

La sección~\ref{sec:estado_del_arte} da una mirada general a como es visto
el problema en la actualidad, destacando algunos trabajos sobresalientes.
En la sección~\ref{sec:ejemplo}, se enuncia una instancia sencilla del problema
a modo de ejemplo. En la sección~\ref{sec:alternativas_modelamiento}, se describe en forma
matemática el problema presentado como ejemplo, dando a conocer las
posibles maneras de modelarlo. \\

Se propone en la sección~\ref{sec:modelo_general} un modelo matemático de programación
lineal mixto para su posterior resolución, sin considerar aún la
rotación de los objetos como posible solución.
Utilizando la herramienta \textit{Lingo}, en la sección~\ref{sec:especificacion},
se formula el modelo general de forma de poder resolver automáticamente el
problema. \\

En la sección~\ref{sec:resolucion}, se muestran resultados obtenidos utilizando el
modelo propuesto y la herramienta \textit{Lindo} para resolver el problema.
En la sección~\ref{sec:perspectivas}, se muestran algunos modelos y técnicas
utilizadas para la resolución de este tipo de problemas.
Finalmente, se exponen en la sección~\ref{sec:conclusiones} algunas de las
conclusiones obtenidas de este trabajo.
