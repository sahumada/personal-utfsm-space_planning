\section{Resolución\label{sec:resolucion}}
Se analizó un caso de prueba muy sencillo debido a la cantidad de restricciones
que se deben escribir a medida que aumentan la cantidad de objetos. Se dispondrá
de tres objetos, dos de ellos de dimensiones $(2, 2, 4)$ y uno de dimensiones $(4, 2, 4)$.\\

El caso de prueba consiste en una superficie $S_1$ de dimensiones
$H=4$, $W=4$ y $D=4$ de alto , ancho y profundidad respectivamente. El
cuadro~\ref{tabla:dos} muestra $3$ objetos que sabe caben en la
supeficie $S_1$. Para resolverlo, sólo se considero minimizar la profundidad
del contenedor; el valor de $M$ se fijó en 1000.

\begin{table}[h]
\begin{center}
\begin{tabular}{||c|c c c||}
\hline\hline
$i$ & $w_i$ & $h_i$ & $d_i$ \\ \hline
1 & 2 & 2 & 4 \\
2 & 2 & 2 & 4 \\
3 & 4 & 2 & 4 \\ \hline\hline
\end{tabular}
\caption{Objetos para analizar el primer caso de prueba}
\label{tabla:dos}
\end{center}
\end{table}

Con estos datos se obtuvo el siguiente modelo de programación lineal, como muestra
el cuadro~\ref{tab:table_1}:\\

\begin{table}[!htb]
\scriptsize
\begin{verbatim}
MIN DT

ST
  ! Mantencion objetos dentro de la superficie
   X_1 <= 2
   X_2 <= 2
   X_3 <= 0
   Y_1 <= 2
   Y_2 <= 2
   Y_3 <= 2
   - DT + Z_1 <= - 4
   - DT + Z_2 <= - 4
   - DT + Z_3 <= - 4
  ! Superposicion de objetos
   - 1000 * P_1_2 - 1000 * Q_1_2 - 1000 * R_1_2 + X_1 - X_2 <= - 2
   - 1000 * P_1_3 - 1000 * Q_1_3 - 1000 * R_1_3 + X_1 - X_3 <= - 2
   - 1000 * P_2_3 - 1000 * Q_2_3 - 1000 * R_2_3 + X_2 - X_3 <= - 2
     1000 * P_1_2 - 1000 * Q_1_2 + 1000 * R_1_2 + X_1 - X_2 >= - 998
     1000 * P_1_3 - 1000 * Q_1_3 + 1000 * R_1_3 + X_1 - X_3 >= - 996
     1000 * P_2_3 - 1000 * Q_2_3 + 1000 * R_2_3 + X_2 - X_3 >= - 996
   - 1000 * P_1_2 - 1000 * Q_1_2 + 1000 * R_1_2 + Y_1 - Y_2 <=   998
   - 1000 * P_1_3 - 1000 * Q_1_3 + 1000 * R_1_3 + Y_1 - Y_3 <=   998
   - 1000 * P_2_3 - 1000 * Q_2_3 + 1000 * R_2_3 + Y_2 - Y_3 <=   998
     1000 * P_1_2 - 1000 * Q_1_2 - 1000 * R_1_2 + Y_1 - Y_2 >= - 1998
     1000 * P_1_3 - 1000 * Q_1_3 - 1000 * R_1_3 + Y_1 - Y_3 >= - 1998
     1000 * P_2_3 - 1000 * Q_2_3 - 1000 * R_2_3 + Y_2 - Y_3 >= - 1998
     1000 * P_1_2 - 1000 * Q_1_2 - 1000 * R_1_2 + Z_1 - Z_2 <=   996
     1000 * P_1_3 - 1000 * Q_1_3 - 1000 * R_1_3 + Z_1 - Z_3 <=   996
     1000 * P_2_3 - 1000 * Q_2_3 - 1000 * R_2_3 + Z_2 - Z_3 <=   996
   - 1000 * P_1_2 + 1000 * Q_1_2 - 1000 * R_1_2 + Z_1 - Z_2 >= - 1996
   - 1000 * P_1_3 + 1000 * Q_1_3 - 1000 * R_1_3 + Z_1 - Z_3 >= - 1996
   - 1000 * P_2_3 + 1000 * Q_2_3 - 1000 * R_2_3 + Z_2 - Z_3 >= - 1996
   ! Restricciones debiles
   P_1_2 + Q_1_2 + R_1_2 <= 2
   P_1_3 + Q_1_3 + R_1_3 <= 2
   P_2_3 + Q_2_3 + R_2_3 <= 2
   P_1_2 + Q_1_2 <= 1
   P_1_3 + Q_1_3 <= 1
   P_2_3 + Q_2_3 <= 1
END
\end{verbatim}
\end{table}
\pagebreak

\begin{table}[!htb]
\scriptsize
\begin{verbatim}
   !Variables binarias
   INT P_1_2
   INT P_1_3
   INT P_2_3
   INT Q_1_3
   INT Q_1_2
   INT Q_2_3
   INT R_1_2
   INT R_1_3
   INT R_2_3
   ! Variables no negativas
   GIN X_1
   GIN X_2
   GIN X_3
   GIN Y_1
   GIN Y_2
   GIN Y_3
   GIN Z_1
   GIN Z_2
   GIN Z_3
\end{verbatim}
\caption{Modelo de programación lineal mixto: Formato Lindo}
\label{tab:table_1}
\end{table}

Luego se procedió a correr el programa llegando al resultado
que muestra a continua\-ción el cuadro~\ref{tab:table_3}: \\

\begin{table}[!htb]
\scriptsize
\begin{verbatim}
 OBJECTIVE FUNCTION VALUE

 1)                 4.000000000

 VARIABLES                      VALUE             REDUCED COST
 DT                       4.000000000              0.000000000
 X_1                      0.000000000              0.000000000
 X_2                      2.000000000              0.000000000
 X_3                      0.000000000              0.000000000
 Y_1                      0.000000000              0.000000000
 Y_2                      0.000000000              0.000000000
 Y_3                      2.000000000              0.000000000
 Z_1                      0.000000000              0.000000000
 Z_2                      0.000000000              0.000000000
 Z_3                      0.000000000              0.000000000
 P_1_2                    0.000000000              0.000000000
 Q_1_2                    0.000000000              0.000000000
 R_1_2                    0.000000000              0.000000000
 P_1_3                    0.000000000              0.000000000
 Q_1_3                    0.000000000              0.000000000
 R_1_3                    1.000000000              0.000000000
 P_2_3                    0.000000000              0.000000000
 Q_2_3                    0.000000000              0.000000000
 R_2_3                    1.000000000              0.000000000
\end{verbatim}
\caption{Resultado del modelo de programación lineal mixto}
\label{tab:table_3}
\end{table}
\normalsize

Se puede notar que el total de los objetos fue posicionado correctamente.
Se realizaron pruebas utilizando distintos casos de prueba y distintos métodos
de resolución, los cuales validarian el modelo para el caso específico que
estamos analizando. \\

La interpretación de los datos resultado nos indica que la profundidad óptima
del contenedor es de 4 unidades; las coordenas del objeto 1 son $(0, 0, 0)$ y
las del objeto 2 son $(0, 2, 0)$ considerando la notación $(x, y, z)$. Además,
la variable $p_{12}$ queda activa, por lo cual se cumple sólo una restricción
de superposición indicando que el objeto 1 esta arriba del objeto 2.
