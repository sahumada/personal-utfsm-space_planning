\section{Ejemplo\label{sec:ejemplo}}
Se propone un ejemplo muy sencillo para ser resuelto posteriormente,
el cual consiste en determinar la mínima altura que debe tener una caja para
empaquetar computadores en su interior. Cada computador tiene tres tipos de piezas:
monitor, cpu y accesorios. \\

Para adecuarnos al problema, se consideró a cada uno de estos tres objetos
con formas rectangulares de dimensiones ($2\times 2\times 2$);
($2\times 1\times 2$) y ($1\times 1\times 1$); de alto, ancho y profundidad
respectivamente. \\

El contenedor consiste en una superficie $S_1$ de dimensiones
$H=4$ y $W=4$ de alto y ancho respectivamente. El cuadro~\ref{tabla:uno}
muestra doce objetos (cuatro computadores, con tres tipos de partes cada
uno) para ser posicionados en la supeficie $S_1$. \\

Para resolver este ejemplo, se considera minimizar la profundidad $D$
del contenedor de manera tal que los doce objetos definidos quepan en $S_1$.

\begin{table}[h]
\begin{center}
\begin{tabular}{||c|c c c||c|c c c||c|c c c||}
\hline\hline
$i$ & $h_i$ & $w_i$ & $d_i$ & $i$ & $h_i$ & $w_i$ & $d_i$ & $i$ & $h_i$ & $w_i$ & $d_i$ \\ \hline
\textbf{1} & 2 & 2 & 2 & \textbf{5} & 2 & 1 & 2 & \textbf{9} & 1 & 1 & 1 \\
\textbf{2} & 2 & 2 & 2 & \textbf{6} & 2 & 1 & 2 & \textbf{10} & 1 & 1 & 1 \\
\textbf{3} & 2 & 2 & 2 & \textbf{7} & 2 & 1 & 2 & \textbf{11} & 1 & 1 & 1 \\
\textbf{4} & 2 & 2 & 2 & \textbf{8} & 2 & 1 & 2 & \textbf{12} & 1 & 1 & 1 \\
\hline\hline
\end{tabular}
\caption{Objetos de instancia sencilla}
\label{tabla:uno}
\end{center}
\end{table}
