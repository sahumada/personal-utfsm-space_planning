\section{Conclusiones\label{sec:conclusiones}}
El problema de planificación de espacio considerado en este trabajo
tiene variadas aplicaciones en la industria. Al ser un problema
$\mathcal{NP}$-duro es muy dificil resolverlo en tiempos acotados a
medida que la cantidad de objetos aumenta. \\

Es por esto que el modelo utilizado es de gran importancia, debido
a que una buena definición de las variables puede incidir de manera
importante en el espacio de búsqueda, el cual es fundamental a la
hora de resolver problemas de este tipo. \\

Las técnicas de resolución también cumplen un papel fundamental cuando
se desea reducir los tiempos de cómputo o se intenta restringir aun más
el problema tratado. Para esto se tienen a disposición las técnicas
completas o incompletas de forma de poder aproximar de buena manera y
obtener un resultado razonable. \\

Las técnicas de enumeración permiten teóricamente
encontrar un óptimo, pero éstas son impracticables cuando superamos un
cierto número de elementos en el dominio de solución. \\

Hemos analizado un caso particular, en el cual el modelo y la herramienta
\textit{Lindo} nos han ayudado a encontrar un valor óptimo para los ejemplos
estudiados y una cota aproximada para el número máximo de objetos que pueden
ser resueltos con ella.
