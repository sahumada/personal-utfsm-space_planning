\section{Perspectivas de investigación\label{sec:perspectivas}}

Para la resolución de este tipo de problemas de optimización, se puede seguir la siguiente
metodología, la cual se esta utilizando en la actualidad por diversos software de
estas características:

\begin{itemize}
\item Presentar los datos del problema en una estructura de datos, para poder manejar
      todas las variables de una manera más simplificada y ordenada.
\item Describir cómo sería la inserción de un cubo dentro del contenedor.
\item Presentar un algoritmo de barrido de un solo paso que muestre todos los candidatos
      a ser insertados, con su correspondiente espacio perdido y su complejidad asociada.
\item Cómo representar los datos obtenidos en una estructura de datos.
\end{itemize}

\begin{enumerate}
\item Se utiliza una forma estándar para conectar el caso de 2 dimensiones con el caso
      de 3 dimensiones, en que el primero será la cara frontal del objeto tridimensional.
      En el frente podemos distinguir vértices, bordes y celdas, de este modo, en cada
      celda, se guarda la profundidad de ella. Esta información se guarda en listas doblemente
      enlazadas separadamente. De esta manera se puede representar el interior de un container
      o de un espacio confinado por medio de una estructura de datos.
\item Al insertar un nuevo objeto (caja), la estructura de datos debe ser modificada, para
      que represente la nueva realidad del frente del contenedor. Esto puede resumirse como
      la inserción de un rectángulo, si lo pensamos como un área bidimensional.
\item En el planteamiento de estos algoritmos podemos distinguir dos tipos fundamentales: \\
Heurísticas:
\begin{itemize}
 \item Constructiva.
 \item Búsqueda local por entornos.
 \item Cadenas de movimiento.
\end{itemize}
Metaheurísticas:
\begin{itemize}
 \item Basado en Búsqueda por entornos: Búsqueda tabú; Búsqueda en entorno variable, VNS; GRASP.
 \item Evolutivos basados en población: Algoritmos genéticos; Algoritmos meméticos; Scatter Search.
\end{itemize}
\item Finalmente, los datos obtenidos deben ser representados en una estructura de datos de
      la misma forma en que fueron definidos inicialmente, a modo de mantener una coherencia
      entre la información inicial y la información final, ya que generalmente esto puede
      verse como una capa dentro de algún software especifico de optimización.
\end{enumerate}
