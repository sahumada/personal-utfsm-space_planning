\section{Estado del Arte\label{sec:estado_del_arte}}

El problema de empaquetamiento o planeación de espacio no sólo esta orientado
a 3 dimensiones, si no que también podemos encontrar problemas de planeamiento
en una y dos dimensiones. Actualmente no existen metodologías de resolución
para este tipo de problemas en un tiempo polinomial, por lo que este tipo
de problemas se clasifican como $\mathcal{NP}$-complejos. \\

Existen, sin embargo, algoritmos como the \textit{First Fit Algorithm}, el cual
posiciona los bloques, u objetos bidimensionales en el orden en que vienen
llegando y busca el primer lograr de calce para cada uno. El problema es que
este algoritmo, difiere del óptimo en aproximadamente un 70\% (Hoffman 1998, p. 171).
Otra estrategia consiste en ordenar primero los objetos desde el más grande
hasta llegar al más pequeño, sin embargo, este método, difiere también del
óptimo, sin embargo, lo hace en aproximadamente un 22\%. \\

Sin embargo, existen modelos más formales para plantear este tipo de problemas,
dadas sus restricciones de espacio y posicionamiento. \\

Por ejemplo, al momento de cortar una plancha de madera para construir una serie
de muebles, es necesario minimizar la cantidad de espacio desperdiciado,
producto de los cortes en la madera, a modo de maximizar la cantidad de
paneles por plancha de madera (Planeación en 2D). \\

Nosotros nos enfocaremos exclusivamente a la planeación en 3 dimensiones, la cual
posee una amplia gama de aplicaciones, como pueden ser el almacenaje de carga en
los aeroplanos, llenado de containeres en forma óptima y, en general, todo lo que
concierne al llenado de espacios tridimensionales en forma óptima. \\

En la actualidad existen software para llevar a cabo diversas tareas de llenado 3D
en forma óptima, como es el caso de 3D Load PackPacker V.1.7x, el cual posee una
base de datos con la información de todos los containeres que utiliza y todos los
tamaños de paquetes para el posible llenado. De esta forma analiza todas las
posibilidades y retorna la que menor espacio perdido tiene. Se ve que este método
de comparaciones es bastante bueno en el caso de formas predefinidas, sin embargo,
esto no es aplicable a casos que no se encuentran en la base de datos del SW en cuestión. \\

Al problema de planeación en 3D se le pueden agregar ciertas modificaciones (Restricciones)
como sería el caso de cargar un container, en el cual existen ciertos paquetes que no
pueden estar con la tapa hacia abajo, otros que tienen que soportar a los que están más
arriba, algunos que no tienen una forma regular, etc. Sin embargo nosotros desarrollaremos
una simplificación de esto, en que todos los paquetes tienen una forma regular y pueden
estar ubicados en cualquier parte del espacio limitado por nuestras restricciones de
posición y ubicación (Traslape).
